\documentclass[a4paper]{article}
\usepackage{listings}

\usepackage{color}
\definecolor{lightgray}{rgb}{.9,.9,.9}
\definecolor{darkgray}{rgb}{.4,.4,.4}
\definecolor{purple}{rgb}{0.65, 0.12, 0.82}

\lstdefinelanguage{JavaScript}{
  keywords={typeof, new, true, false, catch, function, return, null, catch, switch, var, if, in, while, do, else, case, break},
  keywordstyle=\color{blue}\bfseries,
  ndkeywords={class, export, boolean, throw, implements, import, this},
  ndkeywordstyle=\color{darkgray}\bfseries,
  identifierstyle=\color{black},
  sensitive=false,
  comment=[l]{//},
  morecomment=[s]{/*}{*/},
  commentstyle=\color{purple}\ttfamily,
  stringstyle=\color{red}\ttfamily,
  morestring=[b]',
  morestring=[b]"
}

\lstset{
   language=JavaScript,
   backgroundcolor=\color{lightgray},
   extendedchars=true,
   basicstyle=\footnotesize\ttfamily,
   showstringspaces=false,
   showspaces=false,
   numbers=left,
   numberstyle=\footnotesize,
   numbersep=9pt,
   tabsize=2,
   breaklines=true,
   showtabs=false,
   captionpos=b
}


\begin{document}
\section{Working with the JavascriptMVC backend.}
Working with the javascript MVC backend involves making AJAX calls to the dynamic fixtures created for the various models. 
Each such call will receive a response of JSON data, either a single object or a list of objects. These objects describe the different model's data,
and it's through these objects that information is grabbed.

If that makes no sense, don't worry, let's explain it a bit, but first a quick glossary for Fixtures and JSON.

\subsection{Fixtures}
Fixtures are a way to simulate HTTP requests in order to grab information. That means that, for example, if we want to get all the beers, we simulate a \texttt{GET} request to \texttt{/beers}.
JavascriptMVC intercepts this request and executes the relevant code in models/fixtures/fixtures.js.

An example of a fixture, as it exists in fixtures.js, looks like this:
\begin{lstlisting}[language=javascript, frame=single, showstringspaces = false, basicstyle=\small\ttfamily]
"GET /beers" : function(request, response)
{
  // Do stuff with the request, then call response with the relevant information.
}
\end{lstlisting}
This means that when we use AJAX to perform a get request to \texttt{/beers}, we will do the code inside that function. The function will then
return with a response call, which is how information will be handled when the request is completed, as will be shown below.

\subsection{JSON}
JSON is a way to structure data in javascript, essentially it's a more human-readable alternative to XML used by Java and Javascript.
For our purposes, the chief advantage of it is how easy it is to get information out of JSON objects in javascript.

For a JSON object \texttt{json}, any value can be retreived by name. For example, if we want to find the \texttt{description} field of \texttt{json}, 
we just refer to it by its name as such: \texttt{json.description}.

\subsection{AJAX}
With that out of the way, here's a more detailed example of how to make use of the backend, through AJAX calls.

An AJAX call is a call to the function \texttt{\$.ajax()} with a JSON object whose values decide what operation to perform. 
If we want to fetch the inventory from the database using the username \textit{tehoi} and password \textit{tehoi}, we would make the following call:
\begin{lstlisting}[language=javascript, frame=single, showstringspaces = false, basicstyle=\small\ttfamily]
$.ajax(
        {
          type: "GET",
          url: "/beers",
          data:
          {
            username: "tehoi",
            password: "tehoi"
          },
          success: function(json)
          {
            // json.data here is a list of all beers.
          },
          error: function(json)
          {
            // Error information is in json.
          }
        }
      );
\end{lstlisting}
and that's pretty much it. That call will make a GET (see the type: parameter) request to \texttt{/beers} (the url: parameter). 
The username and password are set inside another JSON object named data:, and that's how you pass along any additional information necessary
by the function, such as id numbers and so on. success is a function, with a JSON object as parameter, that is executed if the request succeeds.
It's in this function any code using the requested information goes and the information is accessed through \texttt{json.data}. 
In the above example, \texttt{json.data} contains the list of beers as returned by the database.

There is an exception to the above format, and that is when you wish to perform operations on single-data, such as logging in. When performing 
that kind of request, the fixture is of the format \texttt{/logins/\{username\}}, which means that instead of defining the username in the
data section of the ajax request, it's defined in the url as follows:
\begin{lstlisting}[language=javascript, frame=single, showstringspaces = false, basicstyle=\small\ttfamily, escapechar=\&]
$.ajax(
        {
          type: "GET",
          url: "/logins/tehoi",
          data:
          {
            password: "tehoi"
          }
          &\ldots&
        }
      );
\end{lstlisting}

\subsection{List of fixtures}
\subsubsection{Beers}
\begin{description}
  \item[GET /beers]\hfill\\
    Using an AJAX get request to /beers retrieves the full inventory of available beers. The list is not cleaned up, so result may contain garbage data.\hfill

    The request requires the following data: \hfill \\
    Username\hfill \\
    Password\hfill \\
    
    The response is a list of JSON objects with each element containing the following information:
    beer\_id\hfill\\
    count\hfill\\
    namn\hfill\\
    namn2\hfill\\
    price\hfill\\
    pub\_price\hfill\\
    sbl\_price\hfill

    Example:\hfill\\
    \begin{lstlisting}[language=javascript, frame=single, showstringspaces = false, basicstyle=\small\ttfamily]
$.ajax(
        {
          type: "GET",
          url: "/beers",
          data:
          {
            username: "tehoi",
            password: "tehoi"
          },
          success: function(json)
          {
            console.log(json.data[0].namn); // Print the name of the 
            // first beer to the console.
          },
          error: function(json)
          {
            // Error information is in json.
          }
        }
      );
    \end{lstlisting}
  \item[GET /beers/\{id\}]\hfill\\
    This request returns detailed information about a single beer.

    The required data, beyond id, are:\hfill\\
    username\hfill\\
    password\hfill\\

    The request returns a JSON containing the following:\hfill\\
    alkoholhalt\hfill\\
    argang\hfill\\
    artikelid\hfill\\
    ekologisk\hfill\\
    forpackning\hfill\\
    forslutning\hfill\\
    koscher\hfill\\
    leverantor\hfill\\
    modul\hfill\\
    namn\hfill\\
    namn2\hfill\\
    nr\hfill\\
    prisinklmoms\hfill\\
    prisperliter\hfill\\
    producent\hfill\\
    provadargang\hfill\\
    saljstart\hfill\\
    slutlev\hfill\\
    sortiment\hfill\\
    ursprung\hfill\\
    ursprunglandnamn\hfill\\
    varnummer\hfill\\
    varugrupp\hfill\\
    volymiml\hfill\\

    Example:\hfill\\
    \begin{lstlisting}[language=javascript, frame=single, showstringspaces = false, basicstyle=\small\ttfamily, escapechar=\&]
  var id = 148803;
  $.ajax({
    type:"GET",
    url:"/beers/"+id,
    data:
    {
      username:"jorass",
      password:"jorass"
    },
    &\ldots&
    });
    \end{lstlisting}
    \item[PUT /beers/\{id\}]\hfill\\
      The PUT request updates an item that already exists in the database, that is to say it is how you change the price or amount of a beer, such as when one is sold.

      The put request requires the following data beyond id:\hfill\\
      username\hfill\\
      password\hfill\\
      amount\hfill\\
      price

      The request does not return anything beyond success/failure.

      Example: 
\newpage
    \begin{lstlisting}[language=javascript, frame=single, showstringspaces = false, basicstyle=\small\ttfamily, escapechar=\&]
var id = 148803;
$.ajax({
  type:"PUT",
  url:"/beers/"+id,
  data:
  {
    username:"jorass",
    password:"jorass",
    amount: "0",
    price: "10.90"
  },
  &\ldots&
});
    \end{lstlisting}
\end{description}
\subsubsection{Accounts}
\begin{description}
\item[GET /accounts/\{username\}]\hfill\\
  This request retrieves information about a user.

  The data requiered, beyond a username, is a password.

  The request returns the following information:\hfill\\
  user\_id\hfill\\
  first\_name\hfill\\
  last\_name\hfill\\
  assets

  Example:\hfill\\
    \begin{lstlisting}[language=javascript, frame=single, showstringspaces = false, basicstyle=\small\ttfamily, escapechar=\&]
var user = "jorass";
$.ajax({
  type:"GET",
  url:"/accounts/"+user,
  data:
  {
    password:"jorass",
  },
&\ldots&
  });
    \end{lstlisting}
\item[PUT /accounts/\{user\_id\}]\hfill\\
  Identical to \textbf{PUT /payments/\{user\_id\}}.
\end{description}
\subsubsection{Purchases}
\begin{description}
\item[GET /purchases]\hfill\\
  This request grabs a list of all purchases registered on the server.
  It requires the username and password of an administrator as data.

  The request returns a JSON containing:\hfill\\
  beer\_id \textit{The beer purchased}\hfill\\
  first\_name \hfill\\
  last\_name \textit{The name of the user who placed the purchase}\hfill\\
  namn \textit{The name of the beer}\hfill\\
  namn2 \hfill\\
  price \hfill\\
  timestamp \hfill\\
  transaction\_id\hfill\\
  user\_id\hfill\\
  username
 
  Example:\hfill\\
  \begin{lstlisting}[language=javascript, frame=single, showstringspaces = false, basicstyle=\small\ttfamily, escapechar=\&]
$.ajax({
  type:"GET",
  url:"/purchases",
  data:
  {
    username:"ervtod",
    password:"ervtod",
  },
  &\ldots&
  });
  \end{lstlisting}

  \item[GET /purchases/{username}]\hfill\\
    Similar to above, except it returns only the purchases made by the user in question. Does not require admin privileges.
    The data required, beyond the username, is a password.

    The returned data is the same as above, except the following has been \textbf{removed}:\hfill\\
    first\_name\hfill\\
    last\_name\hfill\\
    username
    
      Example:\hfill\\
  \begin{lstlisting}[language=javascript, frame=single, showstringspaces = false, basicstyle=\small\ttfamily, escapechar=\&]
$.ajax({
  type:"GET",
  url:"/purchases",
  data:
  {
    username:"ervtod",
    password:"ervtod",
  },
  &\ldots&
  });
  \end{lstlisting}
\item[POST /purchases/']
  This adds a single purchase to the database.

  This does not require admin privileges and the data required are as follows:\hfill\\
  username\hfill\\
  password\hfill\\
  beer\_id

  The request returns nothing beyond success/error.

  Example:\hfill\\
  \begin{lstlisting}[language=javascript, frame=single, showstringspaces = false, basicstyle=\small\ttfamily, escapechar=\&]
$.ajax({
  type:"POST",
  url:"/purchases/",
  data:
  {
    username:"jorass",
    password:"jorass",
    beer_id: 14882
  },
  &\ldots&
  });
  \end{lstlisting}
\end{description}
\subsubsection{Payments}
\begin{description}
  \item[GET /payments]\hfill\\
    This request retrieves all payments made by users into their account-assets.
    The request requires an admin's username and password.
    
    The request returns the following data:\hfill\\
    admin\_id \textit{The ID of the administrator who logged the payment}\hfill\\
    admin\_username\hfill\\
    amount\hfill\\
    first\_name \textit{The name of the owner of the account.}\hfill\\
    last\_name \hfill\\
    timestamp \hfill\\
    username \textit{The username of the account being paid into.}\hfill\\

    Example:\hfill\\
    \begin{lstlisting}[language=javascript, frame=single, showstringspaces = false, basicstyle=\small\ttfamily, escapechar=\&]
$.ajax({
  type:"GET",
  url:"/payments",
  data:
  {
    username:"ervtod",
    password:"ervtod"
  },
  &\ldots&
});
    \end{lstlisting}
    \item[GET /payments/\{username\}]\hfill\\
      Like above, but returns payment information only about a single user.
      Does not require admin privileges but does require username and password.
      
      The information returned is as follows:\hfill\\
      admin\_id\hfill\\
      amount\hfill\\
      timestamp\hfill\\
      transaction\_id\hfill\\
      user\_id\hfill\\
  Example:\hfill\\
  \begin{lstlisting}[language=javascript, frame=single, showstringspaces = false, basicstyle=\small\ttfamily, escapechar=\&]
$.ajax({
  type:"GET",
  url:"/payments/ervtod",
  data:
  {
    username:"ervtod",
    password:"ervtod"
  },
  &\ldots&
 });
  \end{lstlisting}     
  \item[PUT /payments/\{user\_id\}]\hfill\\
    This request adds a new payment onto the account of the user with id user\_id.
    This action requires administrator privileges.
    
    The request takes the following data:\hfill\\
    user\_id \textit{The id of the user whos account is to be paid into.}\hfill\\
    username \textit{The username of the admin approving the payment.}\hfill\\
    password \textit{The password of the admin approving the payment.}\hfill\\
    amount

    This request returns nothing but success/failure.
  Example:\hfill\\
  \begin{lstlisting}[language=javascript, frame=single, showstringspaces = false, basicstyle=\small\ttfamily, escapechar=\&]
$.ajax({
  type:"PUT",
  url:"/payments/2",
  data:
  {
    username:"ervtod",
    password:"ervtod",
    amount: 50
  },
  &\ldots&
 });
  \end{lstlisting}   
\end{description}
\subsubsection{Logins}
\begin{description}
  \item[GET /logins/\{username\}]\hfill\\
    This request performs a login operation and returns some basic user information. The request requires the username and password of a non-admin user.

    The request returns the following data:\hfill\\
    user\_id\hfill\\
    username\hfill\\
    password\hfill\\
    first\_name\hfill\\
    last\_naem
    assets

  Example:\hfill\\
  \begin{lstlisting}[language=javascript, frame=single, showstringspaces = false, basicstyle=\small\ttfamily, escapechar=\&]
$.ajax({
  type:"GET",
  url:"/logins/jorass",
  data:
  {
    password:"jorass"
  },
  &\ldots&
 });
  \end{lstlisting}

  \item[GET /logins/admin/\{username\}]
    Same as above but for admins.

    The request returns the same information as above as well as a list of all purchases stored in purchases\_list.

  Example:\hfill\\
  \begin{lstlisting}[language=javascript, frame=single, showstringspaces = false, basicstyle=\small\ttfamily, escapechar=\&]
$.ajax({
  type:"GET",
  url:"/logins/admin/ervtod",
  data:
  {
    password:"ervtod"
  },
  &\ldots&
 });
  \end{lstlisting}    
\end{description}
\end{document}
\documentclass

%  Example:\hfill\\
%  \begin{lstlisting}[language=javascript, frame=single, showstringspaces = false, basicstyle=\small\ttfamily, escapechar=\&]
$.ajax({
  type:"GET",
  url:"/logins/admin/ervtod",
  data:
  {
    password:"ervtod"
  },
%  &\ldots&
% });
%  \end{lstlisting}
